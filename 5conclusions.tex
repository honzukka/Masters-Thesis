\chapter{Conclusions}
\label{chapter:conclusions}

{\color{red} So far, these are only notes which should be updated as I refine the draft. At the end, the conclusion can be written based on them.}

\begin{itemize}
    \item We assume that our projector is 100\% calibrated
        \begin{itemize}
            \item That's fine because our goal is just to improve pure compensation power
            \item But the reader should realize this and keep it in mind throughout reading the thesis!
        \end{itemize}
    \item Current texture models are not perfect, but already good enough to achieve interesting results.
        \begin{itemize}
            \item Likely a good idea to know what exactly some of the drawbacks of these texture models are! (Literature can help, but they can also some from our experiments. Just be specific.)
        \end{itemize}
    \item FUTURE WORK
        \begin{itemize}
            \item How to calibrate (i.e. capture the LT matrix) in real life?
            \item Pushing texture models further
            \item By the way, is this the way to go? Rendering function + texture model? Or can the statistics-based projection mapping pipeline look differently in the future?
            \item Resolution issues
        \end{itemize}
\end{itemize}